\documentclass[polish,12pt,titlepage]{article}

\usepackage{mathtools}
\usepackage{graphicx}
\usepackage{graphics}
\usepackage{amsmath}
\usepackage{amssymb}

\usepackage[utf8]{inputenc}
\usepackage[T1]{fontenc}
\usepackage[polish]{babel}
\selectlanguage{polish}

\usepackage{listings}
\usepackage{url}
\usepackage{path}
\usepackage{here}

\usepackage{color}
\definecolor{lightgray}{gray}{0.6}

%\setlength{\textwidth}{400pt}

\newcommand{\RR}{\mathbb{R}}
\newcommand{\NN}{\mathbb{N}}
\newcommand{\QQ}{\mathbb{Q}}
\newcommand{\ZZ}{\mathbb{Z}}
\newcommand{\TAB}{\hspace{0.50cm}}
\newcommand{\IFF}{\leftrightarrow}
\newcommand{\IMP}{\rightarrow}
\DeclarePairedDelimiter{\floor}{\lfloor}{\rfloor}

\newtheorem{theorem}{Theorem}[section]
\newtheorem{lemma}{Lemma}[section]
\newtheorem{example}{Example}[section]
\newtheorem{corollary}{Corollary}[section]
\newtheorem{definition}{Definition}[section]

\makeindex

\begin{document}

\title{Metody Optymalizacyjne - sprawozdanie z listy 3}
\author{Mateusz Kubuszok (179956)}
\maketitle

\section{Zadanie 1}

Zadanie polega na uszeregowaniu zadań aby czas wykonania był jak najkrótszy. Część zadań nie może być rozpoczęta przez zakończeniem innych zadań, każde może też wymagać wykorzystania zasobów i niemożliwe jest rozpoczęcie zadania, gdy nie ma dość wolnych zasobów do jego wykonania.

Parametrami są:
\begin{itemize}
	\item dodatnia całkowita liczba zadań $n$,
	\item dodatnia całkowita liczba rodzajów zasobów $p$,
	\item dodatnie limity poszczególnych zasobów $Limits_r$, $r \in \{1, ..., p\}$,
	\item nieujemne ilości zasobów wymagane przez każde zadanie $Requires_{j,r}$, $j \in \{1, ..., n\}$, $r \in \{1, ..., p\}$,
	\item czas trwania poszczególnych zadań (l. całkowita dodatnia) $Duration_j$, $j \in \{1, ..., n\}$,
	\item zbiór zależności $(i,j) \in Dependency$, $i,j \in \{1, ..., n\}$ (zadanie $i$ zależy od zadania $j$).
\end{itemize} 

\subsection{Zmienne pomocnicze}

\begin{itemize}
	\item zbiór zadań $Jobs = {1, ..., n}$,
	\item zbiór typów zasobów $Resources = {1, ..., p}$,
	\item maksymalny czas wykonania wszystkich zadań (w pełnych jednostkach czasowych):
	$$MaxExecutionTime = \sum\limits_{j \in Jobs} Duration_j$$
	\item dyskretny zbiór momentów w czasie od rozpoczęcia do maksymalnego czasu wykonania (co 1 jednostkę):
	$$Times = {0, ..., MaxExecutionTime}$$
\end{itemize}

\subsection{Zmienne decyzyjne}

\begin{itemize}
	\item zmienne binarne $Active_{j,t}$ opisujące czy zadanie $j$ zaczęło się w chwili $t$,
	\item zmienne nieujemne $ResourcesInTime_{r,t}$ opisujące ilość wolnych zasobów $r$ w chwili $t$,
	\item nieujemny całkowity czas zakończenia ostatniego zadania $FinishTime$.
\end{itemize}

\subsection{Ograniczenia}

\begin{itemize}
	\item każde zadanie ma tylko jeden moment rozpoczęcia:
	$$\forall_{j \in Jobs} \sum\limits_{t \in Times} Active_{j,t} = 1$$
	\item zadania przestrzegają kolejności:
	$$\forall_{(i,j) \in Dependency} \sum\limits_{t \in Times} t*Active_{j,t}  + Duration_j \leq \sum\limits_{t \in Times} t*Active_{i,t}$$
	\item wartość $FinishTime$ określa czas po zakończeniu wszystkich zadań,
	$$\forall_{(i,j) \in Dependency} \sum\limits_{t \in Times} t*Active_{j,t}  + Duration_j \leq FinishTime$$
	\item ilość zasobów nigdy nie przekracza ustalonego z góry limitu:
	$$\forall_{r \in Resources, t \in Times} ResourcesInTime_{r,t} \leq Limits_r$$
	\item zadania alokują zasoby przy rozpoczęciu i zwalniają przy zakończeniu:
	$$\forall_{r \in Resources} ResourcesInTime_{r,0} = Limits_r - \sum\limits_{j \in Jobs} Active_{j,0} * Requires_{j,r}$$
	$$\forall_{r \in Resources, t \in Times-\{0\}} ResourcesInTime_{r,t} = ResourcesInTime_{r,t-1}$$
	$$ + \sum\limits_{j \in Jobs, t-Duration_j \geq 0} Active_{j,t-Duration_j} * Requires_{j,r}$$
	$$ - \sum\limits_{j \in Jobs} Active_{j,t} * Requires_{j,r}$$
\end{itemize}

\subsection{Funkcja celu}

$$FinishTime \rightarrow minimalizuj$$

\subsection{Wyniki}

\begin{center}
\begin{tabular}{ | c | c | c | }
	\hline
	Zadanie & Start & Koniec \\
	\hline
	1 & 0   & 50  \\
	2 & 96  & 143 \\
	3 & 51  & 106 \\
	4 & 50  & 96  \\
	5 & 143 & 175 \\
	6 & 108 & 165 \\
	7 & 143 & 158 \\
	8 & 175 & 237 \\
	\hline
\end{tabular}
\end{center}

\begin{center}
Czas całkowity: $237$ jednostek czasu.
\end{center}

\begin{center}
Diagram Gantt'a:
\texttt{
j~~~0~~~50~~51~~96~~106~108~143~158~165~175~237~\linebreak
1~~~[~~~]~~~~~~~~~~~~~~~~~~~~~~~~~~~~~~~~~~~~~~~\linebreak
2~~~~~~~~~~~~~~~[~~~~~~~~~~~]~~~~~~~~~~~~~~~~~~~\linebreak
3~~~~~~~~~~~[~~~~~~~]~~~~~~~~~~~~~~~~~~~~~~~~~~~\linebreak
4~~~~~~~[~~~~~~~]~~~~~~~~~~~~~~~~~~~~~~~~~~~~~~~\linebreak
5~~~~~~~~~~~~~~~~~~~~~~~~~~~[~~~~~~~~~~~]~~~~~~~\linebreak
6~~~~~~~~~~~~~~~~~~~~~~~[~~~~~~~~~~~]~~~~~~~~~~~\linebreak
7~~~~~~~~~~~~~~~~~~~~~~~~~~~[~~~]~~~~~~~~~~~~~~~\linebreak
8~~~~~~~~~~~~~~~~~~~~~~~~~~~~~~~~~~~~~~~[~~~]~~~\linebreak
}
\end{center}

\end{document}