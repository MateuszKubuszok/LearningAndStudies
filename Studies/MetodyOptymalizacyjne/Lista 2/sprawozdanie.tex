\documentclass[polish,12pt,titlepage]{article}

\usepackage{mathtools}
\usepackage{graphicx}
\usepackage{graphics}
\usepackage{amsmath}
\usepackage{amssymb}

\usepackage[utf8]{inputenc}
\usepackage[T1]{fontenc}
\usepackage[polish]{babel}
\selectlanguage{polish}

\usepackage{listings}
\usepackage{url}
\usepackage{path}
\usepackage{here}

\usepackage{color}
\definecolor{lightgray}{gray}{0.6}

%\setlength{\textwidth}{400pt}

\newcommand{\RR}{\mathbb{R}}
\newcommand{\NN}{\mathbb{N}}
\newcommand{\QQ}{\mathbb{Q}}
\newcommand{\ZZ}{\mathbb{Z}}
\newcommand{\TAB}{\hspace{0.50cm}}
\newcommand{\IFF}{\leftrightarrow}
\newcommand{\IMP}{\rightarrow}
\DeclarePairedDelimiter{\floor}{\lfloor}{\rfloor}

\newtheorem{theorem}{Theorem}[section]
\newtheorem{lemma}{Lemma}[section]
\newtheorem{example}{Example}[section]
\newtheorem{corollary}{Corollary}[section]
\newtheorem{definition}{Definition}[section]

\makeindex

\begin{document}

\title{Metody Optymalizacyjne - sprawozdanie z listy 2}
\author{Mateusz Kubuszok (179956)}
\maketitle

\section{Zadanie 1}

Tartak produkujący deski szerokości 22 cali otrzymał zamówienie na deski:

\begin{center}
\begin{tabular}{ | c | c | }
	\hline
    szerokość (cale) & ilość (sztuki) \\
    \hline
	7 & 109 \\
	5 & 117 \\
	3 & 79 \\
	\hline
\end{tabular}
\end{center}

Parametrami w zadaniu będą:
\begin{itemize}
    \item liczba $MaxWidth$ określająca maksymalną szerokość desek (tu: 22 cale),
    \item zbiór $Types$ definiujący 3 dopuszczalne warianty desek,
    \item zbiór $TypeWidth_t$, $t \in Types$ definiujący szerokości poszczególnych typów desek,
    \item zbiór $TypeRequired_t$, $t \in Types$ definiujący zamawiane ilości typów desek.
\end{itemize} 

\subsection{Zmienne pomocnicze}

\begin{itemize}
	\item zbiór $TypeMaxAmount_t = \floor{\frac{MaxWidth}{TypeWidth_t}}$, $t \in Types$ definiujący maksymalną ilość desek danego typu jaką możemy otrzymać przez cięcie jednej deski,
	\item zbiór
	$$Amounts = \{(t1,t2,t3):$$
	$$(t1,t2,t3) \in \prod_{t \in Types} \{1, ..., TypeMaxAmount_t\} \and$$
	$$t1*TypeWidth_{T1} + t2*TypeWidth_{T2} + t3*TypeWidth_{T3} \le MaxWidth,$$
	$$t1 + t2 + t3 \ge 1\}$$ definiujący wszystkie możliwe cięcia ($t1$, $t2$ i $t3$ to ilości desek uzyskane z tego cięcia),
	\item zbiór $RestFromDivision_{(t1,t2,t3)} = MaxWidth - (t1*TypeWidth_{T1} + t2*TypeWidth_{T2} + t3*TypeWidth_{T3})$, $(t1,t2,t3) \in Amounts$ definiujący resztki po każdym z możliwych cięć.
\end{itemize}

\subsection{Zmienne decyzyjne}

Zmienne to liczby całkowite nieujemne $DivisionAmount_{(t1,t2,t3)}$, $(t1,t2,t3) \in Amounts$ określające ilości desek pociętych danym sposobem.

\subsection{Ograniczenia}

Należy wyprodukować zamówioną ilość desek danego typu - przykładowo dla typu pierwszego:

\begin{center}
$$\sum\limits_{(t1,t2,t3) \in Amounts} t1 * DivisionAmount_{(t1,t2,t3)} \ge TypeRequired_{T1}$$
\end{center}

Analogicznie dla pozostałych typów desek.

\subsection{Funkcja celu}

Funkcja celu ma za zadanie zminimalizować ilość odpadów:

\begin{center}
$$\sum\limits_{(t1,t2,t3) \in Amounts} DivisionAmount_{(t1,t2,t3)} * RestFromDivision_{(t1,t2,t3)} \longrightarrow minimalizuj$$
\end{center}

Minimalizuje ona ilość resztek z cięć ale nie gwarantuje, że nie pozostaną nam nieużywane deski jednego z 3 zamawianych typów.

\subsection{Wyniki}

Dla danych z zadania otrzymano następujące wyniki:

\begin{tabular}{ | c | c | c | c | }
	\hline
	cięcie & ilość ciętych desek & resztki na cięcie & resztki łącznie \\
	\hline
	1*7", 0*5", 5*3" & 8  & 0 & 0 \\
	1*7", 3*5", 0*3" & 26 & 0 & 0 \\
	2*7", 1*5", 1*3" & 39 & 0 & 0 \\
	\hline
\end{tabular}

\begin{tabular}{ | c | c | c | c | }
	\hline
	typ desek & uzyskana ilość & wymagana ilość & nadprodukcja \\
	\hline
	7" & 112 & 109 & 3 \\
	5" & 117 & 117 & 0 \\
	3" & 79  & 79  & 0 \\
	\hline
\end{tabular}

Całkowita ilość zużytych desek: $73$.

Całkowita ilość resztek: $0$.



\subsection{Interpretacja wyników}

Możliwe jest pocięcie desek tak, aby ilość resztek była równa 0. Jednakże powstaje wówczas nadprodukcja desek 7", i owe 3 nadmiarowe deski należałoby liczyć jako niewykorzystane odpady.

\section{Zadanie 3}

Celem zadania jest uszeregowane procesów na maszynie. Procesy mają ustalony czas przed, którym nie mogą być uruchomione, czas działania oraz wagę.

Parametrami zadnia są:
\begin{itemize}
	\item zbiór procesów $J$,
	\item czasy działania $Duration_j$, $j \in J$,
	\item wagi $Wage_j$, $j \in J$,
	\item czasy gotowości $Readiness_j$, $j \in J$.
\end{itemize}

\subsection{Zmienne pomocnicze}

$$EarliestStart = \min_{j \in J} Readiness_j$$ - definiuje najwcześniejszy możliwy start,
$$MaxExecutionTime = EarliestStart + \sum\limits_{j \in J} Duration_j$$ - definiuje najpóźniejszy możliwy czas zakończenia,
$$LatestStart = MaxExecutionTime - \min_{j \in J} Duration_j$$ - definiuje najpóźniejszy możliwy start.

\subsection{Zmienne decyzyjne}

$EarliestStart \le Time_j \le LatestStart$, $j \in J$ - całkowita nieujemna liczba określająca moment uruchomienia procesu $j$.

$Order_{i,j} \in \{0,1\}$, $i,j \in J$ - zmienna binarna określająca czy proces $i$ jest uruchamiany przed procesem $j$.

\subsection{Ograniczenia}

\begin{itemize}
	\item proces musi być gotowy:
	$$\forall_{j \in J} Time_j \ge Readiness_j$$
	\item z każdych 2 procesów jeden musi rozpocząć się wcześniej:
	$$\forall_{i,j \in J, i \neq j} Order_{i,j} + Order_{j,i} = 1$$
	\item każdy proces kończy się w dopuszczalnym czasie:
	$$\forall_{i,j \in J, i \neq j} Time_i + Duration_i \le Time_j + MaxTotalExecutionTime * Order_{i,j}$$
\end{itemize}

\subsection{Funkcja celu}

$$\sum\limits_{j \in J} Wage_j * Time_j \longrightarrow minimalizuj$$

\subsection{Wyniki}

\begin{tabular}{ | c | c | c | c | c | c | }
	\hline
    Proces & Gotowość & Start & Koniec & Czas pracy & Start - Gotowość \\
    \hline
	1 & 0  & 0  & 5  & 5 & 0  \\
	2 & 3  & 5  & 7  & 2 & 2  \\
	3 & 10 & 20 & 28 & 8 & 10 \\
	4 & 0  & 7  & 13 & 6 & 7  \\
	5 & 2  & 13 & 20 & 7 & 11 \\
	\hline
\end{tabular}

\begin{tabular}{ | c | c | c | }
	\hline
	Proces & Waga  & Koszt \\
	\hline
	1 & 9  & 0      \\
	2 & 4  & 20     \\
	3 & 1  & 20     \\
	4 & 10 & 70     \\
	5 & 2  & 26     \\
	\hline
	Łącznie & & 136 \\
	\hline
\end{tabular}

\subsection{Interpretacja wyników}

Procesy będą uruchamianie w kolejności: 1 $\rightarrow$ 2 $\rightarrow$ 4 $\rightarrow$ 5 $\rightarrow$ 3.


\section{Zadanie 4}

Zadanie polega na najszybszym wykonaniu podanych zadań na 3 maszynach (minimalizacji czasu zakończenia ostatniego zadania).

Parametrami są:
\begin{itemize}
	\item zbiór zadań $j \in J$,
	\item zbiór maszyn wykonujących zadania $m \in M$,
	\item czas działania zadania zadania $j$: $Duration_j$,
	\item binarna zależność między zadaniami $i$ i $j$: $Dependency_{i,j}$ ($i$ zależy od $j$).
\end{itemize}

\subsection{Zmienne pomocnicze}
\begin{itemize}
	\item maksymalny czas wykonania:
	$$MaxCompletionTime = \sum\limits_{j \in J} Duration_j$$
	\item możliwe zależności między zadaniami (zadanie $i$ po zadaniu $j$):
	$$Followings = \left\{ (i,j): (i,j) \in J \times J \land i \neq j \right\}$$
	\item możliwe wykonania zadań na maszynach (zadanie $j$ na maszynie $m$):
	$$Placements = J \times M$$
\end{itemize}

\subsection{Zmienne decyzyjne}
\begin{itemize}
	\item czas rozpoczęcia dla poszczególnych wykonań $(j,m) \in Placements$: $Time_{(j,m)}$,
	\item binarne zmienne określająca czy dane wykonanie $(j,m) \in Placements$ zostało faktycznie użyte: $Executor_{(j,m)}$,
	\item binarne zmienne określająca czy dana zależność $(i,j)$ została wykorzystana na danej maszynie $m$: $Order_{i,j,m}$,
	\item czas wykonywania wszystkich zadań: $Completion$.
\end{itemize}

\subsection{Ograniczenia}
\begin{itemize}
	\item $Completion$ jest nie mniejsze niż czas zakończenia dowolnego zadania:
	$$\forall_{(j,m) \in Placements} Time_{(j,m)} + Duration_{j} \leq Completion$$
	\item każde zadanie jest wykonywane na dokładnie jednej maszynie:
	$$\forall_{j \in J} \sum\limits_{m \in M} Executor_{(j,m)} = 1$$
	\item każde 2 zadania, mogą mieć zdefiniowaną co najwyżej 1 relację kolejności wykonania na pewnej maszynie $m$,
	$$\forall_{(i,j) \in Followings} \sum\limits_{m \in M} Order_{(i,j),m} \leq 1$$
	\item zadania zachowują zdefiniowaną parametrycznie kolejność wykonywania (zadanie $i$ wykonywanie po zadaniu $j$):
	$$\sum\limits_{m_1 \in M} Time_{(i,m_1)} \geq \sum\limits_{m_2 \in M} ( Time_{(j,m_2)} + Duration_j * Executor_{(j,m)})$$
	\item na tej samej maszynie nie mogą wykonywać się 2 zadania w tym samym czasie:
	$$\forall_{(i,j) \in Followings, m \in M}$$
	$$Time_{(i,m)} + MaxExecutionTime *(1-Order_{(i,j),m}) \geq Time_{(j,m)} + Duration_j * Executor_{(j,m)}$$
	$$\land$$
	 $$Time_{(j,m)} + MaxExecutionTime *Order_{(i,j),m} \geq Time_{(i,m)} + Duration_i * Executor_{(i,m)}$$
\end{itemize}

\subsection{Funkcja celu}

$$Completion \rightarrow minimalizuj$$

\subsection{Wyniki}

\begin{center}
\begin{tabular}{ | l | c | c | c | }
	\hline
	Proces & Maszyna & Start & Koniec \\
	\hline
	1 & 3 & 0 & 1 \\
	2 & 1 & 0 & 2 \\
	3 & 3 & 1 & 2 \\
	4 & 1 & 2 & 4 \\
	5 & 3 & 2 & 3 \\
	6 & 1 & 6 & 7 \\
	7 & 3 & 4 & 7 \\ 
	8 & 2 & 3 & 9 \\
	9 & 3 & 7 & 9 \\
	\hline
\end{tabular}
\end{center}

\begin{center}
Czas całkowity: $9$ jednostek.
\end{center}

\begin{center}
Graf wyświetlony na konsoli:\linebreak
\texttt{
M:~~0~~~~~~1~~~~~~2~~~~~~3~~~~~~4~~~~~~5~~~~~~6~~~~~~7~~~~~~8~~~~~~9\linebreak
1:~~[~2~~~~2~~~~~~][~4~~~4~~~~~~]~~~~~~~~~~~~~[~6~~~~]~~~~~~~~~~~~~~\linebreak
2:~~~~~~~~~~~~~~~~~~~~~~~[~8~~~~8~~~~~~8~~~~~~8~~~~~~8~~~~~~8~~~~~~]\linebreak
3:~~[~1~~~~][~3~~~][~5~~~]~~~~~~[~7~~~~7~~~~~~7~~~~~~][~9~~~9~~~~~~]
}
\end{center}

\subsection{Interpretacja wyników}

Na optymalne wykonanie zadań potrzeba 9 jednostek czasu. Przykładowe uszeregowanie i czasu uruchomienia przedstawia graf.

\end{document}