\documentclass[polish,12pt,titlepage]{article}

\usepackage{graphicx}
\usepackage{graphics}
\usepackage{amsmath}
\usepackage{amssymb}

\usepackage[utf8]{inputenc}
\usepackage[T1]{fontenc}
\usepackage[polish]{babel}
\selectlanguage{polish}

\usepackage{listings}
\usepackage{url}
\usepackage{path}
\usepackage{here}

\usepackage{color}
\definecolor{lightgray}{gray}{0.6}

%\setlength{\textwidth}{400pt}

\newcommand{\RR}{\mathbb{R}}
\newcommand{\NN}{\mathbb{N}}
\newcommand{\QQ}{\mathbb{Q}}
\newcommand{\ZZ}{\mathbb{Z}}
\newcommand{\TAB}{\hspace{0.50cm}}
\newcommand{\IFF}{\leftrightarrow}
\newcommand{\IMP}{\rightarrow}

\newtheorem{theorem}{Theorem}[section]
\newtheorem{lemma}{Lemma}[section]
\newtheorem{example}{Example}[section]
\newtheorem{corollary}{Corollary}[section]
\newtheorem{definition}{Definition}[section]

\makeindex

\begin{document}

\title{Metody Optymalizacyjne - sprawozdanie z listy 1}
\author{Mateusz Kubuszok (179956)}
\maketitle

\section{Zadanie 1}

Treścią zadania jest wykonanie testu na odporność solvera na błędy numeryczne. W tym celu wykorzystuje się macierz Hilbera znaną ze słabego uwarunkowania i z tego powodu często używaną w tego typu testach.

Parametrami w zadaniu będą:
\begin{itemize}
    \item liczba $n$ określająca stopień macierzy $A$, oraz
    \item długość wektora $b$ w równaniu liniowym. Określa ona również wielkość wektora zmiennych $x$.
\end{itemize} 

\subsection{Zmienne decyzyjne}

Jedyną zmienna jest wektor $x$. Jego kolejne wartości określają współczynniki o jaki pomnożone zostały kolejne kolumny macierzy hilbertowskiej $A$ przed ich sumowaniem, tak że uzyskano w efekcie wektor $b$.

\subsection{Ograniczenia}

Pierwszym z nałożonych ograniczeń jest wymóg spełnienia równania liniowego $Ax = b$:

\begin{center}
$\forall_{i = 1...n} \sum\limits_{j=1}^n A_{ij}x_{j} = b_{i}$
\end{center}

Kolejnym jest wymóg, aby wszystkie wartości wektora $x$ były dodatnie:

\begin{center}
$\forall_{i = 1...n}{x_i \geq 0}$
\end{center}

Wynikają one wprost z treści zadania.

\subsection{Funkcja celu}

Dla zadania określona została następująca funkcja celu:

\begin{center}
$minimalizuj: f_c(x) = \sum\limits_{i=1}^n b_i x_i$
\end{center}

Przyjmuje ona optymalną wartość dla wektora jedynek $x_i = 1$.

\subsection{Wyniki}

Dla $n = 10$ otrzymano następujący wektor rozwiązań:

\begin{center}
$x = [0.999938530206807, 1.00373274244084,  0.944628406742733, 1.33830530509672, \linebreak
     0 2.39779419248719, 0.521733065497489, 0 2.1745683660549, 0.619265228609009]$
\end{center}
     
zaś wartość funkcji celu była następująca:

\begin{center}
$f_c(x) = 13.3754280633893$
\end{center}

\subsection{Interpretacja wyników}

Prawidłowym rozwiązaniem jest wektor jedynek. Błąd względny (obliczany przy pomocy normy wektora) szacujemy na $0.763195132833894$.

Rozbieżność wynika ze wspomnianego już złego uwarunkowania macierzy Hilbera - ponieważ wynik jest obliczany numerycznie ma to duży wpływ na otrzymane wartości.

\section{Zadanie 2}

Treścią zadania jest znalezienie optymalnych ilości paliwa dostarczanego od $j$ możliwych dostawców do $i$ możliwych lotnisk.

Parametrami są:
\begin{itemize}
    \item deklaracje dostępnych lotnisk oraz dostawców,
    \item koszty dostaw (w dolarach za galon),
    \item zapotrzebowanie poszczególnych lotnisk, oraz
    \item zapasy magazynowe dostawców (w galonach paliwa).
\end{itemize}

\subsection{Zmienne decyzyjne}

Zmiennymi są wartości macierzy $A = [a_{ij}]$,  zawierającej ilość paliwa w galonach dostarczanego do $i$-tego lotniska z $j$-tej firmy.

\subsection{Ograniczenia}

Wymagamy aby każde lotnisko $i$ dostało wystarczającą ilość paliwa:

\begin{center}
$\forall_{i} zapotrzebowanie_i \leq \sum_{j} A_{ij}$
\end{center}

jak również, aby każdy dostawca był w stanie wykonać zamówienie:

\begin{center}
$\forall_{j} zapasy_{j} \geq \sum_{i} A_{ij}$
\end{center}

Wymagamy również, aby ilość dostarczanego paliwa dla każdego lotniska od każdego dostawcy była nieujemna:

\begin{center}
$\forall_{i} \forall_{j} A_{ij} \geq 0$
\end{center}

\subsection{Funkcja celu}

Funkcja celu dana:

\begin{center}
$minimalizuj: f_c(A) = \sum_{i} \sum_{j} A_{ij} * koszt_{ij}$
\end{center}

ma zadanie zminimalizować całkowity koszt zaopatrywania lotnisk.

\subsection{Wyniki}

Dla zdefiniowanych w treści zadania parametrów otrzymano następujące wyniki:
\begin{itemize}
    \item lotnisko 1. kupiło cały potrzebny zapas ($110 000$ galonów paliwa) od 2. dostawcy,
    \item lotnisko 2. kupiło $165 000$ galonów od 1., a $55000$ od 2. dostawcy,
    \item lotnisko 3. kupiło cały potrzebny zapas ($330 000$ galonów paliwa) od 3. dostawcy,
    \item lotnisko 4. kupiło $110 000$ galonów od 1., a $330 000$ od 3. dostawcy,
    \item całkowity koszt wyniósł $8525000$ dolarów.
\end{itemize}

\subsection{Interpretacja wyników}

Dla każdego lotniska najbardziej opłacalne byłoby wybranie najkorzystniejszego dostawcy i zakupienie u niego całego potrzebnego paliwa. Jest to widoczne na przykładzie 1. oraz 3. lotniska.

Jednakże dostawcy nie dysponują nieograniczonymi zapasami. W takim wypadku konieczne jest takie rozłożenie zakupów między kilku dostawców, aby koszt był najmniejszy. Zarówno lotnisko 2. jak i 4. zrealizowały zapotrzebowanie przy pomocy dwóch różnych dostawców.

\section{Zadanie 3}

Treścią zadania jest obliczenie ilości odpowiednich rodzajów ropy jakie na należy kupić, aby w procesach destylacji oraz krakowania katalitycznego otrzymać wymagane ilości paliw ciężkich, domowych oraz silnikowych.

Od paliwa domowego wymaga się dodatkowo nieprzekraczania określonej ilości stężenia siarki.

Parametrami są:
\begin{itemize}
    \item ilości produktów jakie musimy otrzymać,
    \item ceny obu gatunków ropy,
    \item koszty destylacji oraz krakowania, oraz
    \item zawartość siarki w półproduktach powstałych w tych procesach,
    \item wydajność destylacji obu rodzajów ropy i krakowania,
    \item zawartość siarki produktach destylacji oraz krakowania oraz maksymalne stężenie siarki w paliwie domowym.
\end{itemize}

\subsection{Zmienne decyzyjne}

Zmiennymi w zadaniu są:
\begin{itemize}
    \item ilości ropy typu pierwszego oraz drugiego jakie należy zakupić, oraz 
    \item zmienne określające rozdział półproduktów w węzłach produkcji:
    \begin{itemize}
        \item rozdział destylatów między krakowaniem katalitycznym, a mieszaniem z innymi półproduktami w celu otrzymania paliw ciężkich,
        \item rozdział oleju w celu mieszania paliw domowych oraz paliw ciężkich,
        \item pomocnicze zmienne określające jaką część krakowanych półproduktów pochodziła z destylacji określonego rodzaju ropy,
    \end{itemize}
    \item ilość półproduktów powstałych w poszczególnych procesach,
    \item całkowita ilość siarki w oleju domowym oraz całkowita ilość oleju domowego.
\end{itemize}

\subsection{Ograniczenia}

Nałożone są następujące ograniczenia:
\begin{itemize}
    \item ilość produktów destylacji musi się wynikać z parametrów określających jej wydajność:
    \begin{center}
    $\forall_{ropa \in RodzajeRopy} \forall_{polprodukt \in Polprodukty} \; destylacja_{polprodukt,ropa} = ilosc_{ropa} * wydajnosc_{destylacja,polprodukt,ropa} $
    \end{center}
    
    \item suma destylatów przeznaczonych do krakowania oraz na mieszanie paliw ciężkich musi być równa ilości destylatów otrzymanych z ropy:
    \begin{center}
    $\forall_{ropa \in RodzajeRopy} \; destylacja_{destylat,ropa} = destylatDoKrakowania_{ropa} + destylatDoPaliwCiezkich_{ropa}$
    \end{center}

    \item suma olejów przeznaczonych do mieszania paliw domowych oraz paliw ciężkich musi być równa ilość olejów otrzymanych z ropy:
    \begin{center}
    $\forall_{ropa \in RodzajeRopy} \; destylacja_{olej,ropa} = olejDoPaliwDomowych_{ropa} + olejDoPaliwCiezkich_{ropa}$
    \end{center}

    \item ilość produktów krakowania musi wynikać z wydajności procesu:
    \begin{center}
    $\forall_{ropa \in RodzajeRopy} \forall_{polprodukt \in Polprodukty \setminus \{destylat\}} \; krakowanie_{polprodukt,ropa} = destylatDoKrakowania_{ropa} * wydajnosc_{krakowanie,polprodukt,ropa}$
    \end{center}

    \item całkowita ilość produktów krakowania musi być sumą produktów krakowania obu rodzajów ropy:
    \begin{center}
    $\forall_{produkt \in Produkty} \forall_{polprodukt \in Polprodukty} \; (polprodukt = skladnik_{produkt}) \implies (paliwoZKrakowania_{produkt} = \sum_{ropa \in RodzajeRopy} krakowanie_{ropa,polprodukt})$
    \end{center}

    \item musi zostać wyprodukowana wymagana ilość paliwa silnikowego:
    \begin{center}
    $minimumPaliwa_{silnikowy} \leq paliwoZKrakowania_{silnikowe} + \sum_{ropa \in RodzajeRopy} destylacja_{benzyna,ropa}$
    \end{center}

    \item musi zostać wyprodukowana wymagana ilość paliwa domowego:
    \begin{center}
    $minimumPaliwa_{domowe} \leq paliwoZKrakowania_{domowe} + \sum_{ropa \in RodzajeRopy} ( destylacja_{resztki,ropa} + destylatDoPaliwCiezkich_{ropa} + olejDoCiezkiegoOleju_{ropa} )$
    \end{center}

    \item musi zostać wyprodukowana wymagana ilość paliwa ciężkiego:
    \begin{center}
    $minimumPaliwa_{ciezkie} \leq paliwoZKrakowania_{ciezkie} + \sum_{ropa \in RodzajeRopy} ( destylacja_{resztki,ropa} + destylatDoPaliwCiezkich_{ropa} + olejDoPaliwCiezkich_{ropa} )$
    \end{center}

    \item całkowita ilość siarki w oleju domowym jest równa sumie siarki zawartej w przeznaczonych do jego produkcji olejach pochodzących z destylacji i krakowania:
    \begin{center}
    $siarkaWPaliwieDomowym = \sum_{ropa \in RodzajeRopy} ( olejDoPaliwDomowych_{ropa} + stezenieSiarki_{destylacja,ropa} + krakowanie_{olej,ropa} * stezenieSiarki_{krakowanie,ropa} )$
    \end{center}
    
    \item całkowita ilość oleju domowego musi być równa sumie oleju przeznaczonego do paliwa domowego oraz oleju z krakowania:
    \begin{center}
    $iloscPaliwaDomowego = paliwoZKrakowania_{domowe} + \sum_{ropa \in RodzjeRopy} \; olejDoPaliwaDomowego_{ropa}$
    \end{center}
    
    \item stężenie siarki w paliwie domowym nie może być większe niż dopuszczalne:
    \begin{center}
    $maksymalneStezenieSiarki * iloscPaliwaDomowego \geq siarkaWPaliwieDomowym$
    \end{center}
    
    \item wszystkie zmienne są nieujemne.
\end{itemize}

\subsection{Funkcja celu}

Funkcją celu jest koszt produkcji:
\begin{center}
$minimalizuj: f_{c} = \sum_{ropa \in RodzajeRopy}({koszt_{ropy}}_{ropa} * ilosc_{ropa} +  koszt_{destylacja} * ilosc_{ropa} + koszt_{kraking} * {ilosc_{destylat}}_{ropa})$
\end{center}
obliczająca sumę kosztów zakupu ropy, destylacji oraz kosztu krakingu destylatów.

\subsection{Wyniki}

Całość zakupionej ropy ($1026030.36876356$ ton) jest typu pierwszego. Całkowity koszt produkcji wyniósł $1345943600.86768$ dolarów.

\subsection{Interpretacja wyników}

Ropa typu drugiego jest nieopłacalna do zakupu. Wynika to być może z wyższych kosztów zakupu oraz faktu, że efektywność jej destylacji jest mniejsza dla półproduktów częściej wykorzystywanych.

\section{Zadanie 4}

Treścią zadania jest ustalenie najbardziej atrakcyjnego planu zajęć dla studenta. Atrakcyjność mierzona jest na podstawie ocen poszczególnych grup zajęciowych do których może zapisać się student.

Student musi być zapisany do jednej z 4 grup ćwiczeniowych dla każdego z 5 obowiązkowych przedmiotów. Dodatkowo chce znaleźć czas na minimum 1 zajęcia sportowe w tygodniu, zaś zajęć akademickich nie powinny trwać więcej niż 4 godziny dziennie. Zajęcia nie mogą kolidować ze sobą, codziennie powinna istnieć też co najmniej godzinna przerwa na obiad między godziną 12. a 14.

Parametrami zadania są:
\begin{itemize}
    \item listy dostępnych przedmiotów, grup zajęciowych i grup sportowych,
    \item lista dni w których odbywają się zajęcia, lista godzin w podczas których mogą trwać zajęcia (rozłącznych odcinków czasu; nie muszą koniecznie trwać 1 godzinę zegarową),
    \item plan godzinowy kursów (czy dana grupa odbywa zajęcia podczas danej godziny, grupy mogą alokować więcej niż jedną godzinę jednocześnie),
    \item oceny grup z danego przedmiotu,
    \item maksymalna liczba godzin zajęciowych dziennie,
    \item plan godzinowy grup sportowych (analogicznie jak w kursach akademickich),
    \item minimalna liczba zajęć sportowych tygodniowo,
    \item godziny, które mogą być wolne danego dnia, aby otrzymać wymaganą przerwę, maksymalny czas na przerwę danego dnia i wymagana ilość wolnego czasu.
\end{itemize}

Alokacje planu zajęć oraz grup sportowych określa się zmiennymi binarnymi {$1$ jeśli zaalokowano, $0$ w.p.p}.

Należy również rozpatrzyć wersję planu z dodatkowymi ograniczeniami:
\begin{itemize}
    \item wszystkie zajęcia odbywają się w poniedziałek, wtorek i czwartek,
    \item wszystkie grupy mają ranking nie mniejszy niż $5$.
\end{itemize}

\subsection{Zmienne decyzyjne}

Zmiennymi decyzyjnymi są:
\begin{itemize}
    \item zmienne określające, czy zapisano się do danej grupy z danego kursu ($1$ jeśli zapisano się do grupy, $0$ w.p.p.),
    \item zmienne określające, czy zapisano się na dane zajęcia sportowe (j.w.).
\end{itemize}

\subsection{Ograniczenia}

Nałożone są następujące ograniczenia:
\begin{itemize}
    \item z danego przedmiotu można się zapisać tylko do jednej grupy zajęciowej:
    \begin{center}
    $\forall_{przedmiot \in Przedmioty} \; \sum_{grupa \in Grupy} \; zapisany_{przedmiot,grupa} = 1$
    \end{center}
    
    \item godziny zajęć nie kolidują ze sobą:
    \begin{center}
    $\forall_{godzina \in Godziny} \; \sum_{przedmiot \in Przedmioty} \sum_{grupa \in Grupy} (planZajec_{przedmiot,grupa,godzina} * zapisany_{przedmiot,grupa}) + \sum_{grupaSportowa \in GrupySportowe} (planTreningow_{godzina} * zapisanyTreningi_{grupaSportowa}) \leq 1$
    \end{center}
    
    \item zajęcia nie zajmują więcej czasu, niż narzucony limit:
    \begin{center}
    $\forall_{dzien \in DniNauki} \; \sum_{godzina \in Godziny_{dzien}} \sum_{przedmiot \in Przedmioty} \sum_{grupa \in Grupy} \linebreak
    zapisany_{przedmiot,grupa} * czasTrwania_{godzina} \leq maksGodzinZajeciowych$
    \end{center}
    
    \item zachowana jest dzienna ilość przerw w zajęciach:
    \begin{center}
    $\forall_{dzien \in Dni} \; maksWolegoCzasu_{dzien} - \sum_{godziny \in PotencjalnieWolneGodziny_{dzien}} \linebreak
    (\sum_{przedmiot \in Przedmioty} \sum_{grupa \in Grupy} \; planZajec_{przedmiot,grupa,godzina} * czasTrwania_{godzina} + \sum_{grupaSportowa \in GrupySportowe} planTreningow_{grupaSportowa} * czasTrwania_{godzina}) \geq minWolegoCzasu$
    \end{center}
    
    \item zachowana jest wymagana ilość zajęć sportowych:
    \begin{center}
    $\sum_{grupaSportowa \in GrupySportowe} zapisanySport_{grupaSportowa} \geq minZajecSportowych$
    \end{center}
\end{itemize}

Osobno rozpatrujemy plan z dodatkowymi ograniczeniami:
\begin{itemize}
    \item nie odbywamy zajęć poza dozwolonymi dniami:
    \begin{center}
    $\forall_{dzien \in Dni \setminus DozwoloneDni} \; \sum_{godzina \in Godziny_{dzien}} \sum_{przedmiot \in Przedmioty} \sum_{grupa \in Grupy} \linebreak
    planZajec_{przedmiot,grupa,godzina} * zapisany_{przedmiot,grupa} = 0$
    \end{center}
    
    \item wszystkie grupy zajęciowe mają ranking większy lub równy wymaganemu:
    \begin{center}
    $\forall_{przedmiot \in Przedmioty} \forall_{grupa \in Grupy} \; ranking_{przedmiot,grupa} \implies zapisany_{przedmiot,grupa} = 0$
    \end{center}
\end{itemize}

\subsection{Funkcja celu}

Funkcja celu określa maksymalną atrakcyjność planu zajęć:

\begin{center}
$maksymalizuj: f_{c} = \sum_{przedmiot \in Przedmioty} \sum_{grupa \in Grupy} \; zapisany_{przedmiot,grupa} * ranking_{plzedmiot,grupa}$
\end{center}

\subsection{Wyniki}

Dla podstawowej wersji zagadnienia otrzymano następujące wyniki:
\begin{itemize}
    \item student jest zapisany do grup zajęciowych:
    \begin{itemize}
        \item algebra - grupa 3.,
        \item analiza matematyczna - grupa 2.,
        \item fizyka - grupa 4.,
        \item chemia organiczna - grupa 2.,
        \item chemia minerałów - grupa 1.,
        \item treningi - grupa 1.,
    \end{itemize}
    \item wartość planu zajęć wg. ocen z rankingu to $37$.
\end{itemize}

Dla wersji z dodatkowymi obwarowaniami rozwiązania nie znaleziono.

\subsection{Interpretacja wyników}

Możliwa jest znalezienie optymalnego planu tak długo, jak tylko ograniczenia pozwalają na jakikolwiek wybór. Jeśli dla danych parametrów nie istnieje rozwiązanie, musimy zrezygnować z pewnych warunków, aby otrzymać plan zajęć optymalny dla pozostałych warunków.

\end{document}